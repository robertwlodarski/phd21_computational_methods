\subsection*{Questions (a) and (b)}
\colorbox{ForestGreen!25}{ \parbox{\textwidth}{
\textbf{\textit{\underline{Question}}}:
\textit{
    Solve the stationary competitive equilibrium for $A=1$ and $A=1.1$.
}}}\\




%=======================================================================
\subsection*{Questions (g)}
\colorbox{ForestGreen!25}{ \parbox{\textwidth}{
\textbf{\textit{\underline{Question}}}:
\textit{
    Calculate the Gini coefficient for each period on the transition path.
}}}\\

To compute the Gini coefficient, I adapt the following formula:
\begin{equation}
    G = \frac{1}{2\mu} \iint p(x) p(x) \left| x-y\right| \; \mathrm{d}x\mathrm{d}y,
\end{equation}
where $\mu$ is the average asset position of the households.
In my Matlab code, $p\left( \cdot \right)$ comes from \texttt{iMarginalDist} in each iteration:
\begin{subequations}
    \begin{align}
        &p(x) = \texttt{iMarginalDist} = \left(\begin{array}{c}
            \varphi_1 \\
            \varphi_2 \\
            \cdots \\
            \varphi_N
        \end{array} \right) \implies \\
        & \iint p(x) p(y)\; \mathrm{d}x\mathrm{d}y = \iint 
         \left(\begin{array}{c}
            \varphi_1 \\
            \varphi_2 \\
            \cdots \\
            \varphi_N
        \end{array} \right) \times   
        \left(
        \begin{array}{cccc}
            \varphi_1 & \varphi_2 & \cdots & \varphi_4
        \end{array}
        \right)
        \\
        & \iint p(x) p(y)\; \mathrm{d}x\mathrm{d}y = \texttt{sum(sum(iMarginalDist * iMarginalDist'))} 
    \end{align}
\end{subequations}