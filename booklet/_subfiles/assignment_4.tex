\subsection*{Questions (a) through (e)}
\colorbox{ForestGreen!25}{ \parbox{\textwidth}{
\textbf{\textit{\underline{Question}}}:
\textit{
\begin{itemize}
    \item Solve the stationary competitive equilibrium for $A=1$ and $A=1.1$.
    \item Define the transitional competitive equilibrium for the jump in $A$.
    \item Compute the transitional competitive equilibrium.
    \item How long does it take for an economy to converge to the new stationary competitive equilibrium?
\end{itemize}
}}}\\


\begin{figure}[]
\centering
\caption{Transition dynamics.}
\begin{subfigure}{.5\textwidth}
  \centering
  \caption{Shock types.}
  \label{fig:a4_capital_transition}
  \includegraphics[width=\linewidth]{_figures/ShockTypes.png}
\end{subfigure}%
\begin{subfigure}{.5\textwidth}
  \centering
    \caption{Capital dynamics.}
    \label{fig:a4_capital_transition}
  \includegraphics[width=\linewidth]{_figures/CapitalTransitions.png}
\end{subfigure}
\end{figure}

\colorbox{ForestGreen!25}{\textbf{\textit{\underline{Part 1}}} Equilibrium definition: }
Let $\boldsymbol{\Theta_0}$ be the original SRCE distribution, for $A=1$, and $v_{T+1}$ be the new SRCE value function, for $A=1.1$.

\begin{definition}(Transitional competitive equilibrium)
    Given $\boldsymbol{\Theta_0}$ and $v_{T+1}$, $\left(g_{a,t},g_{c,t},v_t, G_t, r_t,w_t,\boldsymbol{\Theta_t}\right)_{t=1}^T$ and
    $\left(g_K,g_{a,L} \right)$ are \textcolor{ForestGreen}{\textbf{transitional compettive equilibrium}} if:
    \begin{enumerate}
        \item $\left(g_{a,t},g_{c,t},v_t\right)_{t=1}^T$ solves the household's problem given $\left(\boldsymbol{\Theta_t} \right)^T_{t=1}$.
        \item $\left(g_K,g_{a,L} \right)$ solves a representative firm's problem given $\left(\boldsymbol{\Theta_t} \right)^T_{t=1}$.
        \item $\left(r_t,w_t\right)_{t=1}^T$ clears the capital and labour markets $\forall t$:
        \begin{subequations}
            \begin{align}
                &\left[K \right]: \quad  g_K \left(\boldsymbol{\Theta_t} \right) = \int a \; \mathrm{d}\boldsymbol{\Theta_t} \\
                &\left[L \right]: \quad  g_{a,L} \left(\boldsymbol{\Theta_t} \right)= \iint \boldsymbol{\Theta_t}(a,z) z \;  \mathrm{d}a\mathrm{d}z.
            \end{align}
        \end{subequations}
        \item The aggregate resource constraint holds:
        \begin{equation}
            \iint g_{c,t}(a,z)+g_{a,t}(a,z)\; \mathrm{d}\boldsymbol{\Theta_t} = F \left(g_K,g_{a,L} \right)+ (1-\delta)g_K.
        \end{equation}
    \end{enumerate}
\end{definition}

%=======================================================================
\subsection*{Questions (g)}
\colorbox{ForestGreen!25}{ \parbox{\textwidth}{
\textbf{\textit{\underline{Question}}}:
\textit{
    Calculate the Gini coefficient for each period on the transition path.
}}}\\

To compute the Gini coefficient, I adapt the following formula:
\begin{equation}
    G = \frac{1}{2\mu} \iint p(x) p(x) \left| x-y\right| \; \mathrm{d}x\mathrm{d}y,
\end{equation}
where $\mu$ is the average asset position of the households.
In my Matlab code, $p\left( \cdot \right)$ comes from \texttt{iMarginalDist} in each iteration:
\begin{subequations}
    \begin{align}
        &p(x) = \texttt{iMarginalDist} = \left(\begin{array}{c}
            \varphi_1 \\
            \varphi_2 \\
            \cdots \\
            \varphi_N
        \end{array} \right) \implies \\
        & \iint p(x) p(y)\; \mathrm{d}x\mathrm{d}y = \iint 
         \boldsymbol{\iota}^T \left(\begin{array}{c}
            \varphi_1 \\
            \varphi_2 \\
            \cdots \\
            \varphi_N
        \end{array} \right) \times   
        \left(
        \begin{array}{cccc}
            \varphi_1 & \varphi_2 & \cdots & \varphi_N
        \end{array}
        \right)\boldsymbol{\iota}\; \mathrm{d}x\mathrm{d}y \implies 
        \\
        & \iint p(x) p(y)\; \mathrm{d}x\mathrm{d}y = \texttt{sum(sum(iMDist * iMDist'))} 
    \end{align}
    which can be combined with the fact that:
    \begin{align}
        & \iint \left| x- y \right| \; \mathrm{d}x\mathrm{d}y= \iint 
        \left|\boldsymbol{\iota}^T \left(\begin{array}{c}
            a_1 \\
            a_2 \\
            \cdots \\
            a_N
        \end{array} \right) - 
         \left(\begin{array}{cccc}
            a_1 & a_2 & \cdots & a_N
        \end{array}\right) \boldsymbol{\iota}
        \right|\; \mathrm{d}x\mathrm{d}y \implies \\
        & \iint \left| x- y \right| \; \mathrm{d}x\mathrm{d}y=\texttt{abs(repmat(vGridA2',pNA2,1)-repmat(vGridA2,1,pNA2))}.
    \end{align} 
    This gives:
    \begin{align}
        & G = \texttt{sum((iMDist*iMDist').*abs(repmat(vGridA2',pNA2,1)-repmat(vGridA2,1,pNA2)),'all')}.
    \end{align}
\end{subequations}